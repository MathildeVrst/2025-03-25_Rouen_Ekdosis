\documentclass[12pt]{book}
\usepackage{fontspec}
\usepackage{geometry} 
\usepackage[classiclatin,french]{babel}
\babelfont{rm}{Old Standard}
\babelfont{sf}{NewComputerModernSans10}
\babelfont{tt}{NewComputerModernMono10}

\usepackage{nextpage}
\usepackage{xltabular}
\usepackage{titlesec}
\titleformat{\section}{\center\Large\bfseries}{}{1em}{}
\titlespacing\section{0pt}{12pt}{0pt} %\titlespacing{command}{left spacing}{before spacing}{after spacing}[right]

\title{Martial. \textit{Épigrammes choisies (Livres I-II)}\footnote{Les textes et traductions ci-dessous sont issues de \textsc{Izaac} H.-J. et \textsc{Malick-Prunier} S., \emph{Martial. Épigrammes} (t. I), Paris, Les Belles Lettres, 2021. Le présent document est un exercice {\eKd} donné dans le cadre de l'atelier \emph{Éditer avec Ekdosis} (Rouen, le 25 mars 2025) assuré par Mathilde Verstraete (qui ne garantit l'élégance du code utilisé).}}
\author{Édité par ...}
\date{}

\usepackage[poetry=verse]{ekdosis}

\FormatDiv{1}{\begin{center}\vspace*{2em}\large\bfseries}{\end{center}}
\FormatDiv{2}{\begin{center}\bfseries}{\end{center}}

\SetAlignment{
  texts=edition[xml:lang="la"];translation[xml:lang="fr"],
  apparatus=edition;translation,
  lineation=page,
  flush=true
}

\AtBeginEnvironment{edition}{\selectlanguage{latin}}
\AtBeginEnvironment{translation}{\selectlanguage{french}\SetLineation{lineation=none}}

\DeclareApparatus{default}[
sep={ : },
subsep={ : }
]

\DeclareApparatus{metrica}[
  sep={\unskip; },
  subsep={\unskip; },
  bhook={\it (\unskip},
  ehook={)},
  rule=\relax
]

\DeclareApparatus{notes}
\SetFootnotes{
type = notes, 
textfnmark = \textsuperscript{#1},
appfnmark = \hskip 1em\textsuperscript{#1}\hskip 0.5em
}


%%%%%Témoins%%%%%

\DeclareWitness{T}{\emph{T}}{Thuaneum florilegium Parisinum 8071}[
msName=Thuaneum florilegium, 
settlement=Paris,
idno=8071,
institution=BNF,
origDate=s. IX-X,
locus=ff. 24\textsuperscript{r}-51\textsuperscript{r}]
\DeclareWitness{R}{\emph{R}}{Vossianum florilegium Leidense Q86}[
msName=Vossianum florilegium, 
settlement=Leyde,
idno=Q86,
institution=Universiteitsbibliotheek Leiden,
origDate=s. IX,
locus=ff. 99\textsuperscript{v}-108\textsuperscript{v}]
\DeclareShorthand{a}{α}{T,R}

\DeclareWitness{P}{\emph{P}}{Palatinus Vaticanus 1696}[
msName=Palatinus Vaticanus, 
settlement=Cité du Vatican,
idno=1696,
institution=Bibliotheca Apostolica Vaticana,
origDate=s. XV]
\DeclareShorthand{b}{β}{P}

\DeclareWitness{A}{\emph{A}}{Vossianus Leidensis Oct. 56}[
msName=Vossianus Leidensis, 
settlement=Leyde,
idno=Oct. 56,
institution=Universiteitsbibliotheek Leiden,
origDate=s. XI]
\DeclareWitness{X}{\emph{X}}{Puteanus Parisinus Lat. 8067}[
msName=Puteanus Parisinus Lat., 
settlement=Paris,
idno=8067,
institution=BNF,
origDate=s. IX uel X]
\DeclareShorthand{c}{γ}{A,X}

\DeclareWitness{n}{\emph{n}}{Florilegium Parisinum Nostradamense}[
msName=Florilegium Nostradamensis, 
settlement=Paris,
idno=17903,
institution=BNF,
origDate=s. XIII,
locus=ff. 63\textsuperscript{v}-70\textsuperscript{v}]


\begin{document}

\maketitle

\cleartooddpage
\section*{Conspectus Siglorum} 

\begin{xltabular}[c]{\linewidth}{lXl}
  \multicolumn{3}{c}{Primae familiae (\getsiglum{a}) codices hi sunt:}\\[0.5em]
  \SigLine{R}\\
  \SigLine{T}\\[1.5em]
  \multicolumn{3}{c}{Secundae familiae (\getsiglum{b}) codices hi sunt:}\\[0.5em]
  \SigLine{P}\\[1.5em]
  \multicolumn{3}{c}{Tertiae familiae (\getsiglum{c}) codices hi sunt:}\\[0.5em]
  \SigLine{A}\\
  \SigLine{X}\\[1.5em]
  \multicolumn{3}{c}{Praeterea duo florilegia hic illic citantur:}\\[0.5em]
  \SigLine{n}\\
  \end{xltabular}

\cleartoevenpage

\section*{Édition \& Traduction}

\SetLineation{
  lineation=none,
  vlineation=page,
  vmodulo,
  vmargin=right
  }  

\begin{alignment}
\begin{edition}
    \ekddiv{head=Liber I, depth=1, n=I, type=book}
\end{edition}
\begin{translation}
    \ekddiv{head=Livre 1, depth=1, n=I, type=book}
\end{translation}
\end{alignment}


\begin{alignment}
  \begin{edition}
    \begin{ekdverse}[type={hexamètres_dactyliques}]
        \ekddiv{head=X, depth=2, n=I.X, type=epigram} 
        \note[type=metrica,nosep,nonum,labelb=I_X_1a,labele=I_X_4h]{Choliambes}
        Petit \app{
    \lem[wit={a,b},alt=gemellus]{Gemellus}
    \rdg[wit=c]{gemellus uenustus} 
}
nuptias Maronillae \\
et cupit et instat et precatur et donat. \\ 
    Adeone pulchra est ? Immo foedius nil est. \\ 
Quid ergo in illa \app{
        \lem[wit={a,c}]{petitur}
        \rdg[wit=b, alt=appe-]{appetitur}
        }
        et placet ? Tussit. \linelabel{I_X_4h}  \\

    \end{ekdverse}
  \end{edition}
  \begin{translation}
    \begin{ekdverse}[type={hexamètres_dactyliques}]
        \ekddiv{head=X, depth=2, n=I.X, type=epigram} 

        Gemellus veut épouser Maronilla. \\ 
        Il la veut, insiste, implore, fait des cadeaux. \\ 
        Elle est si belle ? Pas du tout ! On ne fait pas plus hideux ! \\ 
        Qu'est-ce qui l'attire, alors, et lui plaît tant ? Sa toux\footnote{Gemellus est un \emph{captator} d'héritage.}. \\ 

    \end{ekdverse}
  \end{translation}
\end{alignment}

%%%%%

\begin{alignment}
    \begin{edition}
      \begin{ekdverse}[type={distiques_élégiaques}]
        \ekddiv{head=XXIX, depth=2, n=I.XXIX, type=epigram} 
          \indentpattern{01}
          \begin{patverse*}
          \note[type=metrica,nosep,nonum,labelb=I_XXIX_1a,labele=I_XXIX_4i]{Distiques élégiaques}
          Fama refert nostros te, Fidentine, libellos \\ 
    non aliter populo quam recitare tuos. \\ 
    Si mea uis dici, gratis tibi carmina mittam : \\ 
    si dici tua uis, \app{
    \lem[wit={b,c}]{hoc}
    \rdg[wit=T, alt=\emph{om.}]{}
    }
    eme, ne mea sint. \linelabel{I_XXIX_4i}  \\
      \end{patverse*}
      \end{ekdverse}
    \end{edition}
    \begin{translation}
        \begin{ekdverse}[type={distiques_élégiaques}]
            \ekddiv{head=XXIX, depth=2, n=I.XXIX, type=epigram} 
            \indentpattern{01}
            \begin{patverse*}
        À ce qu'on dit, Fidentinus, \\ 
        tu récites en public \\ 
        mes petits livres, \\
        comme si c'étaient les tiens. \\ 
        Tu veux présenter ces vers comme les miens ? \\ 
        Je t'en enverrai gratis. \\ 
        Tu veux les présenter comme les tiens ? \\ 
        Paie pour les avoir\footnote{Littéralement : paie le fait qu'ils ne soient plus à moi. Il s'agit d'acheter le silence de l'auteur.}. \\
        \end{patverse*} 
      \end{ekdverse}
    \end{translation}
  \end{alignment}

  %%%%%

\begin{alignment}
    \begin{edition}
       \ekddiv{head=LXXI, depth=2, n=I.LXXI, type=epigram} 
          \note[type=metrica,nosep,nonum,labelb=I_LXXI_1a,labele=I_LXXI_4h]{Distiques élégiaques}
          Laeuia sex cyathis, septem Iustina bibatur, 
    quinque \app{
    \lem[wit=c, alt=lycas]{Lycas}
    \rdg[wit=b]{lycis}
    }%
    , Lyde quattuor, Ida tribus. 
    Omnis ab \app{
    \lem[wit=c]{infuso}
    \rdg[wit=b, alt=eff-]{effuso}
    }
    numeretur amica Falerno, 
    et quia nulla uenit, tu mihi, Somne, ueni.\linelabel{I_LXXI_4h} 
    \end{edition}
    \begin{translation}
          \ekddiv{head=LXXI, depth=2, n=I.LXXI, type=epigram} 
                Six coupes pour Laevia, sept pour Iustina, 
                cinq pour Lycas, quatre pour Lydé, trois pour Ida. 
                Que chacune de mes amies soit comptée en rasades de Falerne\footnote{Pour chaque lettre du nom, est versé un cyathe de vin.}, 
                et puisqu'aucune n'est venue, viens sur moi, Sommeil.
    \end{translation}
  \end{alignment}
  
%%%%%
\begin{alignment}
    \begin{edition}
        \ekddiv{head=Liber II, depth=1, n=I, type=book}
    \end{edition}
    \begin{translation}
        \ekddiv{head=Livre 2, depth=1, n=I, type=book}
    \end{translation}
    \end{alignment}
%%%%%

  \begin{alignment}
    \begin{edition}
        \ekddiv{head=VII, depth=2, n=II.VII, type=epigram} 
          Declamas belle, causas agis, Attice, belle ; 
          historias bellas, carmina bella facis ; 
          componis belle mimos, epigrammata belle ; 
          bellus grammaticus, bellus es astrologus,  
          et belle cantas et saltas, Attice, belle ; 
          bellus es arte lyrae, bellus es arte pilae. 
          Nil bene cum facias, facias tamen omnia belle, 
          uis dicam quid sis ? Magnus es ardalio. 
    \end{edition}
    \begin{translation}
            \ekddiv{head=VII, depth=2, n=I.VII, type=epigram} 
                Tu déclames joliment, tu plaides, Atticus, joliment ; 
                tu écris de jolies histoires, de jolis poèmes ; 
                tu composes de jolis mimes, des épigrammes jolies ; 
                tu es joli grammairien, joli astrologue, 
                tu chantes joliment et tu danses, Atticus, joliment ; 
                tu joues joliment de la lyre, joliment à la balle. 
                Tu ne fais rien de bien, mais tu fais tout joliment. 
                Pour un amateur, quel professionnalisme !\footnote{Littéralement: « Veux-tu que je te dise ce que tu es ? Tu es un grand amateur. » \emph{Ardalio} est un terme rare et péjoratif, désignant un dilettante vainement affairé et incapable de mener une activité à son terme.} 
    \end{translation}
  \end{alignment}

%%%%%

\begin{alignment}
    \begin{edition}
        \ekddiv{head=LXIV, depth=2, n=II.LXIV, type=epigram} 
          \note[type=metrica,nosep,nonum,labelb=II_LXIV_1a,labele=II_LXIV_10h]{Distiques élégiaques}
          Dum modo causidicum, dum te modo rhetora fingis 
          et non decernis, \app{
            \lem[wit={a,c}]{Laure}
            \rdg[wit=b]{taure}
          }, quid esse uelis, 
          Peleos et Priami transît et Nestoris aetas 
          et fuerat serum iam tibi desinere. 
          Incipe, tres uno perierunt rhetores anno, 
          si quid habes animi, si quid in arte uales. 
          Si schola damnatur, fora litibus omnia feruent, 
          ipse \app{
            \lem[wit={b,c}]{potest}
            \rdg[wit=T]{potes}
          } fieri Marsua causidicus. 
          Heia age, rumpe moras : quo te \app{
            \lem[wit=b]{sperabimus}
            \rdg[wit={a,c}, alt=-uimus]{sperauimus}
          }
          usque ? 
          Dum \app{
            \lem[wit={a,b}]{quid}
            \rdg[wit={c}]{quis}
          } sis dubitas, iam potes esse nihil. \linelabel{II_LXIV_10h}  
    \end{edition}
    \begin{translation}
            \ekddiv{head=LXIV, depth=2, n=II.LXIV, type=epigram} 
        Pendant que tu t'imaginais tantôt avocat, tantôt rhéteur\footnote{C'est-à-dire professeur de rhétorique.},
        sans te décider, Laurus, pour telle ou telle carrière, 
        se sont écoulées les années de Pélée, de Priam et Nestor, 
        et l'âge de la retraite serait déjà passé. 
        Lance toi ! – trois rhéteurs ont déjà trépassé en un an – 
        si tu as du c\oe ur, si tu as du talent. 
        Tu tires un trait sur l'école ? Les procès embrasent tous les forums, 
        et Marsyas\footnote{Les avocats de Rome se donnaient volontiers rendez-vous près de la statue du satyre Marsyas, sur le forum.} en personne pourrait se faire avocat. 
        Allez, debout ! Plus d'excuses ! Jusqu'à quand faudra-t-il t'attendre ? 
        À balancer sur ce que tu veux être, tu risques de n'être plus rien. 
    \end{translation}
  \end{alignment}

\end{document}